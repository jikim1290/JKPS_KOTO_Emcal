% !TEX root = JKPS_TeX_sample_eps.tex

\section{Introduction}
\label{sec:intro}

A strong momentum correlation of produced particles from relativistic heavy-ion collisions is thought to be originated from an initial geometric anisotropy and hydrodynamic evolution~\cite{Busza:2018rrf}.
Two-particle correlations in \dphi (azimuthal angle difference) and \deta (pseudorapidity difference) have been measured to study azimuthal correlations among produced particles, and a clear correlation has been observed in long-range ($|\deta|>2$) where correlations from resonance decay and jet fragmentation are expected to be small.
Interestingly, such long-range correlation has been also observed in small collision systems like \pp, \pAu, and \pPb collisions over a wide range of collision energy at Relativistic Heavy Ion Collider (RHIC) and the Large Hadron Collider (LHC)~\cite{Nagle:2018nvi}.
Extensive studies have been performed to quantify the collective behavior in small collision systems, and a significant elliptic flow has been measured even in \pp collisions~\cite{Aad:2015gqa,Khachatryan:2016txc}.
Many theoretical approaches have been introduced to explain the data.
One is to apply hydrodynamic evolution developed to describe the flow in heavy-ion collisions~\cite{Weller:2017tsr}, and the other is a model of initial-state correlations among gluons~\cite{Dumitru:2010iy}.
The models considering hydrodynamic behavior generally provide a better description of the data, but it is not yet conclusive.

Recently another model called string shoving model was proposed to describe the long-range correlation in high multiplicity \pp collisions~\cite{Bierlich:2016vgw}.
This model introduces a repulsive force between flux tubes so that the flux tube expands both longitudinally and transversely.
In a high multiplicity \pp event, where many partonic interactions can occur, one string may overlap with many other strings.
The repulsion among these overlapping strings right after the collision results in correlation between particles even in a large \deta range.
In the comparison with the experimental data shown in Ref.~\cite{Bierlich:2017vhg}, this model implemented in \pythia event generator~\cite{Sjostrand:2007gs} shows a qualitative agreement in long-range correlation functions.
For more detailed quantitative comparisons, we perform a full analysis of \pythia events with string shoving and calculate quantities which have been measured in experiments.
In the following sections, we describe the analysis procedure and show results using \pythia event generator with different configurations.
The comparison with experimental results and discussion will be followed.